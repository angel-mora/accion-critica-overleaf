\chapter{El Bloom es el sujeto económico del Imperio}
\label{cha:el-bloom-es-el-sujeto-económico-del-imperio}

Hasta ahora hemos analizado el Estado desde su devenir molar y las operaciones de sus aparatos de captura y modos de producción a través de máquinas sociales. En este capítulo analizaremos el Estado desde el Bloom, una instancia molecular del Estado que bien podríamos señalar como el hombre-masa, para entender como NRx apela al sentir de esta multitud. La razón por la que escogí al hombre-masa es porque es el sujeto económico del Imperio, la forma-sujeto por excelencia, aquel que está atado a, y por el cual tiene sentido, toda la forma mercantil de aparatos y maquinas. El proceso que a escala molar, toma el aspecto del Estado moderno, a escala molecular, se llama sujeto económico. Designa un proceso de monopolización creciente de la violencia legítima, un proceso de deslegitimación de toda violencia excepto la suya. Su existencia obstaculiza cada vez más drásticamente el libre juego de las formas-de-vida, para extraerles nuda vida \autocite{tiqqunIntroduccionGuerraCivil2008}. En el capítulo anterior expliqué la transición del Estado moderno a su repliegue contemporáneo, el Imperio, así como el desarrollo de la cibernética y de la tecnología como parte del proceso de sofisticación e incorporación de los mecanismos del Estado moderno a escalar molecular y molar. Sin embargo, he hablado poco del malestar que produce el ideal de la ciudadanía y cómo los neorreaccionarios son el síntoma de la crisis de la religión de Estado, que impide a cada persona vivir conforme consigo misma sin desear un afuera, es decir, sin desear espectacularmente. En este capítulo explicaré cómo se produce el resentimiento del neorreaccionario que produce el derretimiento (\emph{meltdown}) que el Comité invisible señala como Bloom, una profunda experiencia de alienación y de nihilismo que invade la vida psíquica de toda forma de vida dentro del Estado. Desarrollaré cómo el derretimiento y la experiencia del Bloom están relacionados con una búsqueda religiosa más allá de la falsa promesa del Estado de totalidad de sentido sobre lo real.

El trabajo es una de las cuestiones clave en el desarrollo de la subjetividad nihilista, específicamente de la ciudadanía. En los siguientes apartados explicaré el papel de la ética del trabajo de la religión de Estado en la configuración de subjetividad del Bloom, una forma masculina que, como veremos, incorpora la falsa conciencia ilustrada. A continuación analizaremos el desarrollo molecular del Estado moderno. Es decir, al sujeto económico.

\begin{table}[htb]
  \caption{Leyenda}
  \label{tab:tablename}
  \centering
  \begin{tabular}{cc}
    \toprule
    \textbf{Instancias del Estado} & \textbf{Escala}\tabularnewline
    \midrule
    Estado moderno & Molar\tabularnewline
    Sujeto económico & Molecular\tabularnewline
    \bottomrule
  \end{tabular}
\end{table}

\section{El rostro contemporáneo de la ciudadanía es el Bloom u hombre masa}
\label{sec:el-rostro-contemporáneo}

El argumento central de este capítulo es que el Estado produce al déspota como representación a desear. Es decir, el deseo de deseo de Estado es también una cierta iconografía, un manual de etiqueta de lo que el déspota hace, de cómo se relaciona con la servidumbre. Lo resumo de la siguiente manera:

\begin{quote}
  \emph{El deseo de deseo de Estado es el deseo del deseo del déspota. Bloom también es apellido del hombre de Estado que vive profundamente frustrado.}
\end{quote}

El Imperio es \emph{kat-echon}: poder histórico que llega a retener el advenimiento del Anticristo y el fin del \emph{eón} actual, como lo señala Carl Schmitt en \emph{El nomos de la Tierra}. El Imperio se aprehende como el último bastión contra la irrupción del caos y actúa dentro de esta perspectiva mínima. El Imperio es la agonía de la civilización. Frente a esta agonía, el Partido Imaginario ocurre cuando el Afuera ha pasado dentro. Es la otra cara del repliegue, en un mundo donde no hay ningún Afuera visible, donde la Gran Locura clásica, la Naturaleza pura o el Gran Proletariado clásico de los obreros pierden toda fuerza de atracción imaginaria. Esto se debe a que hay afuera por todas partes, en cada punto del tejido biopolítico. El Biopoder no rige directamente sobre los hombres o cosas, sino sobre posibilidades y condiciones de posibilidad.

El hombre masa que se vuelve segmentado vive una forma de alienación de forclusión de su potencia gaudendi, de la abyección de sus propios deseos en el espejo de sus pulsiones. Nunca le es posible reconocerse pues no hay un otro, solo hay mimesis del algoritmo que transforma el gusto en capital-clic. El clic es la forma contemporánea del \emph{purchase}, de la información en metadatos producida por una transacción económica. Este poder de configuración performativa y molecular se desenvuelve en circuitos de excitación-frustración como lo define Preciado es profundamente adictivo \autocite{preciadoTestoYonqui2008}, \autocite{anderssonSocialMediaDeliberately2018}. El sentir del Bloom es de aburrimiento. Patrones de conducta del déspota de dominación y servidumbre como codigos programados en procesos exhaustivos de capacitación empresarial y coaching. PNL es un ejemplo.

Adam Curtis muestra en \emph{Century of the Self} la forma como las corporaciones utilizaron técnicas de asociación y practicas psicoanalíticas para dar forma a las relaciones públicas. El libro \emph{Propaganda} de Edward Bernesse fue el primer panfleto de manipulación de masas. En la actualidad estas técnicas se han refinado y sofisticado al grado de llegar a volverse algorítmicas, con un algoritmo responsivo y adaptativo que interactúa con las reacciones de los usuarios. La publicidad instaura el deber, el sujeto de Estado no desea, debe.

\subsection[En el trabajo \emph{sujeta} en las dimensiones\ldots{}]%
{En el trabajo sujeta en las dimensiones libidinal-molecular e industrial-molar del Imperio}
\label{sub:en-el-trabajo-sujeta-en}

\epigraph{\enquote{Cuando el esclavo descontento coge jovialmente el brazo de su
señor, hace sentir la fuerza que tendría su revuelta}}{\emph{Peter Sloterdijk} en~\autocite{sloterdijkCriticaRazonCinica2014}}

Un punto importante para comprender la transformación del Estado-nación al Imperio se encuentra en el papel del trabajo. Por ello, en este apartado analizaremos las condiciones de producción del trabajo del hombre-masa en la era cibernética. Para comenzar, tenemos que recordar que la dominación mercantil tiende a expandir sus dominios a toda área de la vida y al hacerlo vuelve trabajo a cualquier acción sujeta de la explotación. Este proceso está relacionado con la transformación de la subjetividad. No por nada el concepto central de las revoluciones de los siglos XIX y XX es la masa, una subjetividad que experimenta la realidad de la misma manera, a través del \emph{consumo estandarizado}. En este proceso, las vivencias en su forma de experiencias cognitivas dan sentido y estructura a la imaginación, es decir a la máquina deseante que es cada singularidad.

Las imágenes crean vivencias y así se configura una idea del pasado, por ejemplo, a través de Hollywood que nos enseña a desear melancólica o espectacularmente) y del futuro, como el telos apocalíptico de ciudades hiperdesarrolladas basadas en una economía tecnocapitalista. A través de estos circuitos de producción tecno-material es que la explotación se extiende a cada rincón del planeta, deja a su paso desolación y muerte para después reconstruir la \enquote{sociedad} en otro momento histórico, como ocurrió después de Auschwitz, a través de los medios que producen para las grandes audiencias, para la industria cultural.\footnote{Que dan forma a la cultura popular y al espectador televisivo en un circuito que lo posiciona como ente pasivo sentado en un sillón consumiendo comida chatarra.} En términos concretos, la masa es el resultado de un proceso sintético en el que el individuo afronta una situación externa a él, participa en la situación y proyecta la situación en otros individuos que habitan el mismo espacio. Como ejemplo está Disney, que transmite efectivamente el deseo de casta a través de sus figuras de princesas, reyes y caballeros. Para profundizar sobre estos puntos conviene revisar los documentales de Adam Curtis, particularmente recomendamos \emph{The Century of the Self} como genealogía de las formas contemporáneas de dominación e \emph{Hypernormalisation}.

Yo he intentado perfilar a la subjetividad ideal del Estado moderno como un sujeto económico que \emph{debe} ser algo parecido al \emph{ciudadano soldado consumidor espectador.} Solo basta recordar que el antecedente histórico del ciudadano han sido los súbditos, los fieles. Quizá desde ahí se perfila el proceso donde las multitudes devienen siempre masa a través de los aparatos de captura. El lenguaje cotidiano es muy útil para hacernos ver cómo se transmite la deuda en la subjetividad de ciudadano soldado consumidor espectador:

\emph{paga tus impuestos}

\emph{sirve a la patria}

\emph{no te pierdas el descuento\ldots{}}

\emph{ni el siguiente show.}

En esta fórmula todas las personas deben. El deber y la deuda provienen del mismo sentir (y de la misma locución latina \emph{debere}). Sin embargo, la deuda tiene una condición que ha sido entregada antes del nacimiento, como una suerte de fruto por el que hay que pagar con el pecado original durante el resto de nuestras vidas. El ciudadano \emph{debe} pagar sus \emph{impuestos}, el soldado \emph{debe} honrar a la Patria, el consumidor \emph{debe} comprar y el espectador \emph{debe} mirar. La deuda es impuesta a través de la norma, una práctica de consumo y de posiciones constante que difumina la decisión para plantear comportamientos en función de roles de género, de clase, raciales, principalmente. La imagen representacional juega aquí un papel fundamental, pues es parte de la economía que re-produce la jerarquía que da forma y sentido al aparato de captura para capacitar a los agentes que operan la maquina capitalista.

\subsection{Bloom, es decir, el ciudadano soldado consumidor espectador está sujeto a través de la pantalla}
\label{sub:bloom-es-decir-el-ciudadano-soldado-consumidor-espectador-está-sujeto-a-través-de-la-pantalla}

La subjetividad propia del Estado moderno es el ciudadano soldado consumidor espectador. Produce votos, paga impuestos, da vida a la sociedad a través de las mercancías, es parte del ejército de reserva y consume imágenes. El Bloom es el sujeto ideal del Imperio. En cierto sentido, señala Tiqqun, la deconstrucción es una nueva forma de policía, es una continuación de la razón pública, de toda crítica como crítica literaria. Es una simple reacción bloomesca porque deja de ver en lo que se lee algo que pudiera relacionarse con su vida. En cambio, ve en lo que vive un tejido de referencias a lo que ha leído ya. La presencia y el mundo en su conjunto, en la medida en que el Imperio le concede los medios para eso, adquieren para él un carácter de pura hipótesis.

Hay, sin embargo, algo de militante en la deconstrucción, como una militancia de la ausencia, una retirada ofensiva en el mundo cerrado pero indefinidamente recombinable de las significaciones. Y tiene una función política precisa, la de hacer pasar por bárbaro a todo lo que se imponga violentamente al Imperio, por místico a quien tome su presencia como centro de energía de su revuelta, por fascista a cualquier consecuencia vívida del pensamiento, cualquier gesto. El enemigo está presente en todos lados bajo la forma del riesgo. La diferencia entre la policía y la población se ha abolido, pues cada ciudadano puede revelarse como policía. Al comprender esta paranoia resulta evidente cómo la huella digital, el residuo de nuestras operaciones, se revela crucial al ser en todo momento utilizable. Que este registro esté disponible hace de cada gesto una amenaza suficiente.

Por otra parte, el algoritmo capitalista nos produce como si se tratara de una cartografía psíquica. El algoritmo del virus capitalista y el condicionamiento de desarrollo de las tecnologías mediáticas para producir excedentes (y residuos) configuran lo que se espera de las personas individualmente y a gran escala. Los medios configuran nuestra capacidad de practicar nuestro gusto o de devenir personas-masa. Tanto así que si en algún momento hubo revoluciones burguesas fue en parte por la capacidad de \emph{leer la prensa escrita} como criterio de consumo literario común suficiente para dar forma a una identidad colectiva revolucionaria, a una identidad de clase. Con el paso del tiempo, la radio y el cine reconfiguraron el potencial revolucionario de las comunicaciones y nuevas formas de sublevación-represión (o gestión) surgieron. \footnote{Por ejemplo, el radio creó un espacio informacional nuevo por el alcance de las emisiones sonoras y la forma como se reproduce. Este cambio tecnológico transforma el modo de concebir el espacio y lo público. Los situacionistas tienen el concepto de \emph{psicogeografía} para referirse al desarrollo urbano y a sus consecuencias en nuestras formas de desear.} Es necesario preguntarnos hoy en día cómo la cibernética ha transformado el potencial revolucionario de los medios y una cuestión central para entenderlo son los memes.

La transformación de la comunicación por la inclusión de nuevas tecnologías nos muestra cómo una forma mediática representa poder y nos revela el papel clave de los \emph{medios,} pues ellos configuran el espacio a través de flujos de comunicación. La configuración (\emph{setup}) de estas redes de comunicación son un catalizador para el cambio social. Consideremos lo siguiente: en la actualidad, los modelos de masa donde un grupo recibe una sola transmisión son reemplazados por modelos donde el individuo recibe una transmisión única gracias a \emph{algoritmos reactivos} que alteran la secuencia del contenido de las redes sociales y de ese modo individualiza y hace única la experiencia de consumo de cultura. La subjetividad ya no es producida como individuos en serie sino a través de segmentos. En el posfordismo, los medios hacen más efectiva la economía del deseo y el algoritmo que reacciona se torna una suerte de mímesis, de espejo.

Este cambio de paradigma de modelos de gobernanza en masa a modelos en red obedece al desarrollo cibernético del algoritmo. Las sociedades disciplinarias y de control son demasiado complejas para gestionar, por lo que es más fácil fijar protocolos para que se autogestionen redes de manera más eficiente. El espacio masivo está condicionado al número de participantes en un espacio y un momento particulares mientras que el espacio de redes se extiende y contrae en el espacio-tiempo de acuerdo a las órdenes y necesidades de la red. Es decir, su uso del capital es más eficiente porque se ajusta a las necesidades de cada momento~\autocite{PeopleArePeople2018}.

La identidad se conforma a través de una relación mediática que produce subjetividades diferenciadas no a partir de sus estructuras psíquicas y de sus deseos sino de sus intereses. El público, la audiencia, la sociedad, la multitud, se configura mediáticamente. Se comunica a través de ciertas estructuras: impresos, altavoces, pantallas, micrófonos, símbolos, gestos, ruidos, memes, etc. Lo que el capitalismo hace es disponer de estos medios para la apropiación de plusvalor. Por esta razón, toda forma de sublevación pasa por una lógica mediática de reapropiación del valor, pues nada revolucionario puede haber si de su base material se alimentan propietarios, banqueros o patrones. En ese sentido, la neorreaccion del whiteness no es otra cosa más que la contradicción entre la idea de libertad y el conflicto que le produce al Bloom reconocerse como agente del capital. Esa tensión que \autocite{huiUnhappyConsciousnessNeoreactionaries2017}, citando a Hegel, identifica como falsa conciencia ilustrada, como dialéctica entre espíritu estoico y escéptico, es también la tensión del reconocimiento tecnomaterial.

El Imperio se representa a sí mismo con gusto, como una red de la cual cada uno sería un nudo. La norma constituye el elemento de la conductividad social en cada nudo. En ese sentido, cada individuo se vuelve un dispositivo. Antes que la información, lo que circula es la causalidad biopolítica, con mayor o menor resistencia, según el grado de normalidad. Las campañas de sensibilización del Imperio son creaciones de fenómenos y de entramados de causalidades que permiten materializarlo en la sensibilidad social. El principal enemigo del Imperio es el acontecimiento, todo lo que podría pasar y que hace peligrar el entramado de normas y dispositivos. De modo que la soberanía imperial consiste en que ningún elemento del espacio-tiempo, del tejido biopolítico, esté a resguardo de su intervención. Bajo esta lógica, los medios de producción tienden a volverse medios de control, pues el edificio jurídico, reducido a un simple arsenal de la norma, hace de todos y cada uno un sospechoso. Por eso, para mí:

\begin{quote}
  \emph{El Bloom es el sujeto económico del Imperio y la neorreacción su ética.}
\end{quote}

En todo caso, el ciudadano es quien está condicionado a la sumisión en las diferentes esferas de lo social (sic. aparato de captura). Reproduce la deuda que da sentido al modo de producción, el valor en sí, una abyección mercantil que hoy en día son promesas sobre promesas en bonos, \emph{hedgefunds} y cuestiones del estilo. La neorreacción es peligrosa porque sus partidarios, agentes de Edipo, pertenecen a la clase que está en el tope de la pirámide de la jerarquía social. La clase angelical, asexuada, de hombres amantes de la Ilustración y tomadores de decisiones, jugando el rol completo de la ciudadanía, haciendo uso de su plena potencia. Fuerza vital de la ausencia, máquina social de mundos tecnomateriales, con complejos y ahora líquidos mecanismos y aparatos, armas y herramientas, de producción de deseo, son los agentes directos, la \emph{intelligentsia} del Imperio. Consumidores y espectadores de la farmacopornografía. Aquello por lo que lo ausente tiene \emph{sentido}. Son cínicos llenos de amargura por el desencanto de la conciencia de sí. La no-incorporación de totalidad de sentido en la conciencia absoluta. Paranoicos cuyo cuerpo les es otro. Para profundizar, en el siguiente apartado desarrollaré el concepto de cinismo como falsa conciencia ilustrada y haré un breve análisis de su relación con la Ilustración.

\subsection{La falsa conciencia ilustrada, pero triste, del neorreaccionario, revela la crisis de la religión de Estado}
\label{la-falsa-conciencia-ilustrada}

\epigraph{\enquote{Si la consciencia simple exige finalmente la disolución de todo este mundo de perversión, resulta que esa consciencia no puede exigir al \emph{individuo} que se aleje de ese mundo [\ldots] esa exigencia hecha al individuo singular es precisamente lo que pasa por el mal [\ldots] pues el \emph{mal} consiste en preocuparse \emph{de sí mismo} en cuanto \emph{singular} [\ldots]. La exigencia de tal disolución sólo puede dirigirse al espíritu mismo de la cultura}}{\emph{El Comité Invisible} en~\autocite{tiqqunTesisSobrePartido}.}

El neorreaccionario es la falsa conciencia ilustrada. Adquiere e incorpora la crítica de la sociedad de consumo, del Espectáculo y de su miseria. Es un cínico y su conciencia aguda se enorgullece de su perfecta impotencia para cambiarlo, movilizada de manera maniaca contra la conciencia de sí y contra toda búsqueda de sustancialidad. En el límite de la sociedad mercantil, esta se instala en el nihilismo para continuar la lógica del estado de excepción. Y el Bloom es diferente del neorreaccionario en cuanto a que su afección es el nihilismo, lo indispone. Lo que vuelve un neorreaccionario al Bloom no es otra cosa sino que su disposición espiritual toma partido por el nihilismo, por la guerra total contra sí mismo.

El \emph{pathos} del neorreaccionario es cínico. Los cínicos son los dueños del mundo y al mismo tiempo, en su calidad de ser sujetos reales totales, son el modelo de ciudadanía. En palabras de Preciado, su deseo es ser violados, transgredidos. Experimentan lo real siempre como simulacro y lo producen como verdad para la representación de sí mismos: el ciudadano soldado consumidor espectador. En este hecho descansa la contradictoria paradoja de nuestra investigación.

Bajo una cultura policiaca, de la represión y de la vigilancia constante, es decir, bajo los albores del régimen disciplinario, se cocinó una entidad política, la sociedad, que no tiene otro fin más que reproducir, en la medida de lo posible, la lógica necropolítica que alimenta al Estado como instauración definitiva de la monopolización de la violencia. Los ilustrados opinan: En todo lugar donde impera la violencia hay un montón de subjetividades idiotas que no saben lo que necesitan ni lo que desean.

La sociedad implementa dispositivos de sentido que instauran el reino de la necesidad sobre el deseo como lógicas desbarbarizantes. En la Ilustración, la crítica tiene la función de reafirmar el poder del Estado y genera una actitud conformista, como la razón pública de Kant. La crítica significa siempre un conformismo. Sloterdijk comenta que \enquote{el cinismo moderno es el resultado de la crítica ilustrada que al desvelar los supuestos de la llamada falsa conciencia ilustrada termina ella misma por volverse la falsa conciencia ilustrada}. Es decir, la Ilustración, con el giro copernicano de la crítica kantiana, no hace sino reafirmar la metafísica de sujeto-objeto, donde el sujeto es una sustancia total, esencial. La modernidad es ruptura con el saber como cuidado de sí y búsqueda de la buena vida. \enquote{Se produce entonces un vacío donde el sujeto, a pesar de conocer y usar el mundo como medio para sus fines, no se encuentra en él} \autocite{anonimoQuinismoImposibleAcerca2010}. La modernidad desliga lo racional de lo subjetivo. La autodeterminación es borrada por la autopreservación.

La ironía de la razón moderna consiste en que debe controlar la fuerza omniabarcante y estratégica emanada de su propia reflexión. En función de lo anterior, podemos entender al Estado en relación con el cinismo, como la experiencia fáctica y mundana como experiencia propia del sujeto que es desterrada del saber ilustrado y moderno si no está mediada por alguna construcción cientificista o en un discurso universal dotador de sentido. Esto es, sin una cosa como la objetividad, ese afuera que da forma y sentido al proyecto de antropología positiva. De este modo, el pathos se rompe y la \emph{areté} es reemplazada por la praxis. La ironía es el lenguaje del cinismo, la posición burlesca y desinteresada de una clase que desea espectacularmente. La Historia es la suma de narrativas que la publicidad, antesala del Espectáculo, usa para enseñar a desear al público, que es siempre potencia revolucionaria, pero cuyo deseo ha sido embrujado porque, al ser voluntad sobre una imagen dada, comparte con la mercancía el carácter fantasmagórico.

La imagen, como la mercancía, adquiere cierta mística que subjetiva a quien la mira, que hace un giro radical en el que el medio es agente de producción de significados y la conciencia solo una instancia sobre la que vaciar cualquier semántica antropologizante. Así, el nihilismo del espectador no es otra cosa que la conquista definitiva del espíritu, la configuración de una audiencia proletaria trasnacional que desea negarse. El cinismo como forma de vida de la metrópoli es resultado de la ironía de volver a ponerse la máscara que pretendió arrancar al Antiguo Régimen. Sin embargo, el cinismo moderno no es de una máscara identificable sino que es difuso. El cinismo supone un rompimiento entre la reflexión y la existencia, ya que el dato empírico que da dominio sobre la objetividad toma el lugar del saber para la vida \autocite{onfrayCinismosRetratoFilosofos2005}. El orden de la filosofía tradicional provenía de una reflexión por la propia fuerza vital y no en una exterioridad como totalidad plena de sentido. \enquote{Cuando el esclavo descontento coge jovialmente el brazo de su señor, hace sentir la fuerza que tendría su revuelta} \autocite{sloterdijkCriticaRazonCinica2014}. Esta condición transforma toda insurrección en un movimiento de interés, es decir, en un momento económico. Todo se vuelve intersticios de caprichos, insolencias secretas y ventajas propias al amparo de la ley. El cinismo de los de abajo muestra la cuestión del sometimiento por placer, una suerte de adaptación somática al miedo y al dolor.

En realidad, el Bloom se parece mucho al cínico, es un individuo amargado, un tipo de masas. La popularización de la crítica lo afirma como sentimiento. El saber del cínico es instrumentalizado y se vuelve una justificación de todo lo justificable. Ni siquiera el individuo confía en sí mismo pues sabe que detrás de él se oculta un tramposo. Para el cínico, el gozo de la vida se suple por la inquietud \emph{esquizofrénica} que se quiere conocer para dominar y controlar. Se resuelve contra todo positivismo al considerarlo como engaño sin darse cuenta de que el positivismo está asentado en la existencia misma. Se vuelve un monólogo racional y objetivante, se estatuye como una \enquote{doctrina inmoral de inteligencia}, niega rotundamente la vida y la posibilidad al individuo de elegirse un totalmente otro. En última instancia, el cinismo se trata de una configuración particular de la subjetividad, un habitus que deambula entre una profunda necesidad de verdad metafísica y un deseo de Espectáculo. Resulta paradójico que, para mucha gente, aunque el régimen esté mal, no existe ni la más remota posibilidad de pensar en derrocarlo, al menos de desconocer su autoridad. O al menos hay algo, una desobediencia, que \emph{podría ser y no es}. La gente internaliza el Estado. Por miedo. Por tradición. Por temor a lo desconocido. En última instancia, porque la crisis, el Espectáculo, es incapacidad de imaginar. En relación con el Estado moderno, el cinismo es el síntoma de la imposibilidad del Estado de constituir una realidad total, la producción ideal del sujeto de Estado.

Lo que el Bloom \emph{debe} hacer es asumir su rol como agente pero ese hecho le produce un vacío pues niega su deseo y le produce un malestar al concebir la libertad como concepto pero nunca como experiencia. De ese modo su espíritu no puede concebir otra cosa que el resentimiento. La fantasía neoreaccionaria de una utopía tecnocapitalista sería un estadio posterior al Imperio, donde todas las contradicciones del Estado son, finalmente, perfectamente asumidas y asimiladas, para dar forma a una nueva forma de regulación biopolítica en donde las barreras entre biopoder y contrato social han sido totalmente desvanecidas. Es decir, se trata de una sociedad donde la libertad en puro concepto y toda forma de experiencia es causa de pena. Esta condición es descrita en los textos de \#altwoke como ultraedipo, la realización concreta de la transformación de la pulsión en deber puro, en norma plenamente interiorizada.

A diferencia de los déspotas, los neorreaccionarios carecen de verdad metafísica y el Espectáculo los ha excluido porque el whiteness significa la des sexualización y el principio del Espectáculo es la gestión de la excitación. De modo que el neorreaccionario es el Señor, aquel que solo puede relacionarse sexualmente a través del interés. Aquel que no puede manifestar su deseo. Quisiera recalcar la cuestión que señalé en el capítulo 2 sobre el deseo del déspota como una forma paranoica. De ahí que cualquier acontecimiento afuera de la norma produzca una fuerte reacción por parte del Estado. Los Bloom son quienes toman partido por la impotencia, quienes in-corporan la abyección de la impotencia. En ese sentido, la religión de Estado es un culto a la frustración y a la represión, a la supresión de toda potencia vital. Sin embargo, existe de hecho una potencia revolucionaria en el hombre-masa en los momentos en los que experimenta plenamente la alienación a la que se ve sometido, un sentimiento de extrañeza donde lo ajeno deja de pasar desapercibido y se incorpora. Tiqqun señala que de esta potencia surge una fuerza capaz de irrumpir violentamente el orden social. Se trata, sin embargo, de una fuerza destructiva, una línea de fuga en la violencia que permite al sujeto, como una olla a presión, liberar un \emph{tanatos}, el residuo vital que acompaña a la producción a través del trabajo alienado. Se trata, siguiendo a Deleuze y Guattari, de la derrama de anti producción necesaria para reproducir el ciclo del capital.

El Bloom es un sujeto cínico cuyo único deseo es experimentar un acontecimiento, cualquier muestra de vitalidad que, sin embargo, necesitan rechazar con la misma fuerza con que la desean. Su cuerpo no deja de ser texto, crítica, discurso sobre la sociedad. La posibilidad de vivir, de hallarse fuera del régimen de enunciación (es decir, de la necesidad de enunciar, de nombrar, como la paranoia edípica), les está completamente negada. En ese sentido, su experiencia del otro es siempre un racionalismo, siempre está afuera del cuerpo. La experiencia del Bloom en cuanto cínico, esta movilizada de manera maniaca contra la conciencia de sí y contra toda búsqueda de sustancialidad. Los Bloom son la potencia del déspota y los neorreaccionarios son los déspotas ilustrados del futuros que desean conducir la religión de Estado hacia el exterminio.

La mística de la religión de Estado es el deseo compulsivo de metafísica, de unidad, de un cuerpo plenamente \emph{organ}izado. La apuesta religiosa de NRx para el Bloom (amor a la nada más allá de clase, etc., por el reconocimiento de la subjetividad) es el discurso tecnocientífico productivista que lleva a la utopía tecnocapitalista, el Ultraedipo. Bloom es la condición del desolado, de quien ya no cree en la religión de Estado. Los NRx toman partido por una alternativa a un Estado moderno que es un Estado con un proyecto universalista y hegemónico en donde no existe el otro, una igualdad en escenario abundancia-jerarquía completamente pacificado \autocite{fraseFourFutures2011,fraseFourFuturesVisions2016}. El escenario después del Imperio, después del exterminio de la diferencia y la instauración efectiva del reino de lo mismo.

La neorreacción juega en la misma experiencia de \emph{Choose your own adventure} que la política folk, una incompletitud con el espíritu absoluto en la Historia. La fantasía tecno política padece de la visión divina de la caja negra, de la incapacidad de ver cosas en las mercancías. La interconexión digital mediatizada del Capital. Ese es el \emph{pathos} neorreaccionario. Mientras tanto, la militancia se ve enfrascada en la fantasía de la sagrada familia, de la construcción política mercantil y estamentaria, basada en roles con manuales de etiqueta aprendidos performativa y mediáticamente. Para mí, el problema del Comité Invisible es que no es capaz de posicionarse frente al espíritu cínico más que con cierto virilismo. Sin embargo, una cuestión interesante es que las condiciones del espectáculo hacen, al mismo tiempo, más capaz espiritualmente al Bloom. La conclusión de este apartado respecto al cinismo:

\begin{quote}
  \emph{El cinismo es el síntoma de la imposibilidad del Estado de constituir una realidad total, la producción ideal del sujeto de Estado, la manifestación de lo que en Tiqqun se enuncia como Bloom.}
\end{quote}

Pese al espíritu negativo del cínico, hoy hay más herramientas para resolver esos problemas que dan tanta fuerza a los embrujos mercantiles. La profanación de esa utopía tecnocapitalista sería una heterotopía de soberanía tecno política, que solo se puede lograr a través de la \emph{tecnocrítica}. A este concepto dedicaremos el siguiente capítulo. Necesitamos un afuera a la jerarquía familiar que solo es la fábrica de deseo que reproduce la sociedad.